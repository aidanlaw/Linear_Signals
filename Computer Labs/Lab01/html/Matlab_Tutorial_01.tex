
% This LaTeX was auto-generated from MATLAB code.
% To make changes, update the MATLAB code and republish this document.

\documentclass{article}
\usepackage{graphicx}
\usepackage{color}

\sloppy
\definecolor{lightgray}{gray}{0.5}
\setlength{\parindent}{0pt}

\begin{document}

    
    
\section*{MATLAB}


\subsection*{Contents}

\begin{itemize}
\setlength{\itemsep}{-1ex}
   \item Session 1
   \item Session 2
   \item Question 4
   \item Question 5
\end{itemize}


\subsection*{Session 1}

\begin{verbatim}
8+3*5 %23
8+(3*5)
(8+3)*5
3*4^2-5
(3*4)^2-5

%Question 4
%a
6*(10/13)+18/(5*7)+5*(9^2)

%b
6*(35^(1/4))+14^(-0.35)

%Question 5
%imaginary numbers
(-5+9i)+(6-2i)
x=-5+9i
y=6-2j
z=x+y

a=x-y
b=x*y
c=x/y

%Question 6
%Complex numbers and trig
mag=abs(x)
phase=angle(x)
phased=phase*180/pi
tan(phase)
tand(phased)

%NOTE: e is not a predefined constant like i,j,pi.
e=exp(1)
exp(i*pi/6)
cos(pi/6)
sin(pi/6)

%Question 7
%Matrices
X=[1 2 3]   %matrix on one row
X=[ 1;
    2;
    3]   %matrix in one column
X=X'        %switch from column to row
Y=[ 3 3 3]
Z1=X+Y
%Z2=X*Y      %multiplication of matrices
Z3=X.*Y     %dot product of matrices

%solve the matrix
% x+3x2+2x3=1
% 2x1+2x2+4x3=2
% 4x1+x2+5x3=5

% Ax=b
A=[ 1 3 2 ;
    2 2 4 ;
    4 1 5 ]

B=[ 1 2 5 ]
C=A / B
\end{verbatim}

        \color{lightgray} \begin{verbatim}
ans =

    23


ans =

    23


ans =

    55


ans =

    43


ans =

   139


ans =

  410.1297


ans =

   14.9909


ans =

   1.0000 + 7.0000i


x =

  -5.0000 + 9.0000i


y =

   6.0000 - 2.0000i


z =

   1.0000 + 7.0000i


a =

 -11.0000 +11.0000i


b =

 -12.0000 +64.0000i


c =

  -1.2000 + 1.1000i


mag =

   10.2956


phase =

    2.0779


phased =

  119.0546


ans =

   -1.8000


ans =

   -1.8000


e =

    2.7183


ans =

   0.8660 + 0.5000i


ans =

    0.8660


ans =

    0.5000


X =

     1     2     3


X =

     1
     2
     3


X =

     1     2     3


Y =

     3     3     3


Z1 =

     4     5     6


Z3 =

     3     6     9


A =

     1     3     2
     2     2     4
     4     1     5


B =

     1     2     5


C =

    0.5667
    0.8667
    1.0333

\end{verbatim} \color{black}
    

\subsection*{Session 2}

\begin{par}
Question 3
\end{par} \vspace{1em}
\begin{par}
Multiple plots, maxima, minima and comparators: For t = 0 to 8 and each each signal s1 = 5sint, s2 = 2?t and s3 = 0.4?(1.8t)
\end{par} \vspace{1em}
\begin{par}
a)    Plot the 3 signals on the same time axes,       use: figure, hold, plot(t, s\# , ?colour letter? ) and/or       plot(t,[s1; s2; s3])
\end{par} \vspace{1em}
\begin{verbatim}
t = 0: .01 : 8; % t=linspace(0,8,1000)
y1 = 5*sin(t);
y2=sqrt(t); %y2=t^0.5;
y3=0.4*(1.8*t).^0.5; %y3=0.4*sqrt(1.8*t);
figure
plot(t,y1); hold on; plot(t,y2);
plot(t,y3)
figure
plot(t,[y1;y2;y3])
% min max straightforward
hold on
plot(t,[y1;y2;y3]>2)% binary output 0 false, 1 true
plot(t,[y1;y2;y3].*([t;t;t]>=2))%turn on at t>=2;
\end{verbatim}

\includegraphics [width=4in]{Matlab_Tutorial_01_01.eps}

\includegraphics [width=4in]{Matlab_Tutorial_01_02.eps}
\begin{par}
b)    \ensuremath{>}\ensuremath{>}max(s\#) , min(s\#) \% Confirm the maximum and minimum values
\end{par} \vspace{1em}
\begin{par}
c)    \ensuremath{>}\ensuremath{>}plot(t , s\#\ensuremath{>}=2) and explain the output
\end{par} \vspace{1em}
\begin{par}
d)    \ensuremath{>}\ensuremath{>}plot(t , s\#.*(t\ensuremath{>}=2)) and explain the output
\end{par} \vspace{1em}


\subsection*{Question 4}

\begin{par}
The roots of a polynomial f(x) are the values of x, such that f(x) = 0. Obtain the roots of the following polynomials:
\end{par} \vspace{1em}
\begin{par}
a)     $$ x^3-4.5x^2+5x-1.5=0 $$
\end{par} \vspace{1em}
\begin{verbatim}
F1=[1 -4.5 5 -1.5];
root=roots(F1) %3 real roots (0.5, 1, 3)
\end{verbatim}

        \color{lightgray} \begin{verbatim}
root =

    3.0000
    1.0000
    0.5000

\end{verbatim} \color{black}
    \begin{par}
b)     $$ x^3-7x^2+40x-34=0 $$
\end{par} \vspace{1em}
\begin{verbatim}
F2=[1 -7 40 -34];
root=roots(F2) %2 complex roots (1, 3+- 5i)
\end{verbatim}

        \color{lightgray} \begin{verbatim}
root =

   3.0000 + 5.0000i
   3.0000 - 5.0000i
   1.0000 + 0.0000i

\end{verbatim} \color{black}
    

\subsection*{Question 5}

\begin{par}
Plot the above polynomials to confirm if the roots were located correctly by
\end{par} \vspace{1em}
\begin{par}
a)    calculating $$ f(x) $$ using array operators for x=[-10:0.2:10]; then plot(x,f)
\end{par} \vspace{1em}
\begin{verbatim}
x= -10:.2:10 ;
f1=x.^3 - 4.5*x.^2 + 5*x -1.5;
f2=x.^3 - 7*x.^2 + 40*x -34;
\end{verbatim}
\begin{par}
b)    using polyval(), e.g. plot(x,polyval([1 -4.5 5 -1.5], x))
\end{par} \vspace{1em}
\begin{verbatim}
figure
plot(x,[f1 ; f2])
figure
plot(x,[polyval(F1,x);polyval(F2,x)])
F=[3 2 -100 2 -7 90];
root=roots(F)
x=linspace(-6,6,1000);
plot(x,polyval(F,x));
\end{verbatim}

        \color{lightgray} \begin{verbatim}
root =

  -6.1423 + 0.0000i
   5.4298 + 0.0000i
   0.9630 + 0.0000i
  -0.4586 + 0.8507i
  -0.4586 - 0.8507i

\end{verbatim} \color{black}
    
\includegraphics [width=4in]{Matlab_Tutorial_01_03.eps}

\includegraphics [width=4in]{Matlab_Tutorial_01_04.eps}



\end{document}
    
